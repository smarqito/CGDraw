\documentclass[relatorio.tex]{subfiles}
\begin{document}

\subsection{Animação com Transformações Geométricas}
A implementação das transformações geométricas podem ser divididas em dois tipos:
\begin{itemize}
    \item Translação
    \item Rotação
\end{itemize}
Para auxiliar a sua implementação, foi criado uma classe que permite calcular 
a taxa de atualização da renderização.
Este valor é essencial para ambos os tipos de transformação.

Especificamente, consiste em ter uma variável que armazene o tempo total que 
a transformação deve acontecer.
\begin{code}
\captionof{listing}{Class TimeControl}
\label{code:time_control}
\inputminted[firstline=6, lastline=15]{cpp}{../../TimeControl.h}
\end{code}

Assim, pode-se inicializar com o tempo total de transformação ou
definir-se um tempo posteriormente.
A cada iteração será executado o \mintinline{cpp}{updateRate}
para manter o $\_refresh\_rate$ atualizado.

\begin{code}
\captionof{listing}{Atualização da taxa de atualização}
\label{code:update_rate}
\inputminted[firstline=22, lastline=25]{cpp}{../../TimeControl.cpp}
\end{code}


\subsubsection{Translação} \label{subsub:transl}
Implementação realizada através da utilização de duas classes auxiliares:
$Curve$ e $TranslateCurve$.
A primeira permite o cálculo da posição e da sua derivada.
Este, é efetuado com recurso a um vetor de pontos de controlo
e utiliza-se a estratégia de escolher um conjunto de 4 pontos
que correspondem aos adjacentes do local em que o objeto se encontra.
Depois de escolhidos os 4 pontos, é populada uma matriz $4x3$ com os valores 
$x,y,z$ dos mesmos.
O cálculo efetivo consiste em multiplicar a matriz de transformação
(predefinida é a Catmull-Roll) pela matriz de pontos criada.
Após calcular o vetor $\overrightarrow{t}$ e $\overrightarrow{d}$,
este é multiplicado pela resultado do cálculo anterior,
originando os valores da posição e da derivada.

A classe $Curve$ é utilizada pela $TranslateCurve$ para calcular 
a posição e a derivada num determinado instante de tempo.
Para tal, utilizou-se as fórmulas de produto externo entre vetores e 
a sua normalização.
Posteriormente, é construído uma matriz de rotação para ser aplicado pelo 
$glMultMatrixf$.

O instante de tempo é calculado com recurso à classe $TimeControl$.

Para facilitar a visualização das curvas de translação, acrescentou-se 
a interação com o utilizador através das teclas 'u' e 'U'.
A primeira, 'u', ativa a renderização da curva de posição, i.e.
o local por onde o objeto vai passar.
A segunda, 'U', vai ativar a renderização das linhas de derivada em cada posição.

Por último, apesar das únicas matrizes de transformação utilizadas neste projeto corresponderem
às pré-definidas, é possível utilizar matrizes definidas pelo utilizador 
para serem aplicadas às curvas de translação.
Assim, é permitido definir qualquer uma das curvas default através 
do acréscimo do atributo \textit{curve="0|1|2"} ou personalizadas
através do código \ref{code:matrix_custom}.
\begin{code}
\captionof{listing}{XML Exemplo com matriz personalizada}
\label{code:matrix_custom}
\begin{minted}{xml}
<translate time = "10" align="True" curve="-1"> 
<matrix m="4" n="4">
    <elem>-0.5</elem>
    <elem>1.5</elem>
    <elem>-1.5</elem>
    <elem>0.5</elem>
    <elem>1.0</elem>
    <elem>-2.4</elem>
    <elem>10</elem>
...
</matrix>
\end{minted}
\end{code}

\subsubsection{Rotação} \label{subsub:rot}
A aplicação da rotação sobre um determinado eixo, foi conseguida através do cálculo 
do instante de tempo (cf. $TimeControl$) e a sua multiplicação por 
$360\deg$.

\subsection{VBOs} \label{subsec:VBO}
No seguimento de terem sido aplicados \textit{VBOs} sem índices na fase II, a implementação dos 
modelos com o uso de índices era a próxima evolução lógica. 
Tinha sido previsto progredir para a implementação de índices.
No entanto, tal funcionalidade será implementada na fase $IV$.
\end{document}
