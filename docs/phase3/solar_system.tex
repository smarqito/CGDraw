\documentclass[relatorio.tex]{subfiles}

\begin{document}
    
\subsection{Modelo do Sistema Solar} \label{subsec:sistema_solar}

O sistema solar estático anteriormente contruído 
foi atualizado, implementando o movimento de rotação do Sol 
e dos Planetas à sua volta, bem como dos seus satélites naturais. 

\subsubsection{Movimento de Translação}
O movimento de translação(de cada planeta à volta do Sol) foi feito com a primitiva 
\textit{rotate} com um vetor unitário centrado na origem, que representa o centro do Sol, a apontar para $y=1$.

Neste sentido,
\begin{code}
\captionof{listing}{Movimento de Translação da Terra.}
\label{code:translate_terra}
\begin{minted}{xml}
<transform>
    <rotate time="36.5" x="0" y="1" z="0" />
</transform>
\end{minted}
\end{code}

\subsubsection{Movimento de Rotação}
O movimento de rotação(de cada planeta à sua volta), também é feito com o \textit{rotate} 
porém com o vetor centrado no próprio astro, na mesma a apontar para $y=1$. Os satélites naturais de 
Terra e Marte, possuem também movimento de rotação e de translação sobre o planeta em que orbitam, 
sendo estes feitos de forma análoga ao movimento dos planetas. 
É importante realçar que para tornar o sistema solar o mais perto da realidade, todos os tempos
dos movimentos utilizados tem por base os valores reais, aplicando apenas uma escala nestes com 
o intuito de melhorar a experiência visual do sistema solar.

Deste modo, a escala usada:
\begin{eqnarray} 
    TempoRot_{x_{modelo}} = (\frac{TempoRot{x_{real}}}{(10*60*24)}) secs \\
    TempoTransl_{x_{modelo}} = (\frac{TempoTransl{x_{real}}}{(60*24)}) secs 
\end{eqnarray}
\dots sendo os valores em segundos.

\subsubsection{Cometa}
No que toca ao movimento do cometa, o grupo inspirou-se no movimento traçado pela trajetória do cometa \textit{Halley},
uma vez que é o mais conhecido do sistema solar. 
Para representar a sua órbita \textbf{elíptica}, foram escolhidos 4 pontos que definem um \textbf{losango} e o centro
deste foi escolhido de maneira a tentar respeitar o seu movimento, 
que num período de tempo está muito próximo do Sol e depois permanece afastado durante muito tempo. 
Desta forma, na \textit{demo-scene} foi utilizada a translação com os 4 pontos que definem uma curva 
\textit{Catmull-Rom}, que é precedida por um \textit{rotate} com um ângulo de $17.7graus$, de maneira a aproximar
a trajetória da rota real do cometa. 
Rota esta que será seguida por um teapot desenhado com base nos
\textit{Bezier patches} fornecidos pela equipa docente.

Assim, os pontos usados:
\begin{code}
\captionof{listing}{Pontos usados para trajetória do cometa.}
\label{code:cometa_curva}
\begin{minted}{xml}
<transform>
<rotate angle="17.7" x="1" y="0" z="0" />
<translate time = "60" align="True" >
        <point x = "0" y = "0" z = "1400" />
        <point x = "-3000" y = "0" z = "-4400" />
        <point x = "0" y = "0" z = "-10200" />
        <point x = "3000" y = "0" z = "-4400" />
</translate>
<scale x="30" y="30" z="30" />
</transform>
\end{minted}
\end{code}
\dots onde é possível verificar os 4 pontos de controlo usados para construir a curva de \textit{Catmull-Rom}, 
bem como a rotação para obter o ângulo desejado.
Neste caso, o parâmetro \textit{Align} encontra-se a \textit{True}, pelo que o cometa encontra-se alinhado 
com a curva na sua trajetória.

\subsubsection{Anéis de Saturno}
Uma animação para o anéis de Saturno foi adicionada, para apresentar o \textbf{aninhamento de transformações}.
Neste caso, são apresentadas como três rotações para permitir os anéis rodarem à volta do planeta.
Do modo como a engine encontra-se desenvolvida, as transformações vão ser aplicadas 
pela ordem que aparecem, sendo permitida esta dinâmica; no caso as rotações estão a ser efetuadas por
tempos e ao longo de vetores diferentes.

O excerto do ficheiro \textit{.XML} do sistema solar:
\begin{code}
\captionof{listing}{Rotação dos anéis de Saturno}
\label{code:saturno_aneis}
\begin{minted}{xml}
<group>
    <transform>
        <rotate angle="-45" x="1" y="0.45" z="1" />
        <rotate time="3" x="0" y="1" z="0" />
        <rotate time="4" x="0.5" y="0" z="0" />
        <rotate time="10" x="0" y="0" z="0.5" />
    </transform>
    <models>
        <!-- Ring A -->
        <model file="ring_a.3d" />
        <!--generator torus 179 10 15 5 ring_a.3d -->
        <!-- Ring B -->
        <model file="ring_b.3d" />
        <!-- generator torus 199 10 15 5 ring_b.3d -->
    </models>
</group>
\end{minted}
\end{code}
\dots sendo que, neste caso, as transformações são aplicadas aos dois \textbf{tórus}, pelo 
que ambos rodam do mesmo modo. 

\end{document}