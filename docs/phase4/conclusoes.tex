\documentclass[relatorio.tex]{subfiles}
\begin{document}
\section{Conclusões}
Como consequência do esforço do grupo ao longo do projeto, a implementação de novas funcionalidades
permanece facilitada dado à arquitetura desenvolvida, i.e. a modularidade do código 
traduz-se na fácil inserção de novos módulos ou parâmetros sem afetar a estrutura final.
A otimização de ambos programas (\textit{generator} e \textit{engine}) mantém-se um
dos objetivos principais do grupo, pelo que esta organização estrutural do projeto é essencial.
Não obstante todos os objetivos a atingir no desenvolvimento deste relatório, manteve-se 
a certeza de que o consumo de memória não crescesse de modo incontrolável; algo que 
ocorreu no decorrer da presente fase, nomeadamente no desenho de curvas.

Analogamente às fases anteriores, foi possível implemetar todos os requisitos propostos 
no enunciado, com a adição de alguns parâmetros extra.
É de notar nesta fase:
\begin{itemize}
    \item Criação de um módulo para a construção de superfícies de \textit{Bezier}.
    \item Rotação com base num dado tempo.
    \item Translação por uma curva, dados o tempo e os pontos de controlo.
    \item Translação com base nos pontos de controlo definidos pelo utilizador, numa matriz.
    \item Transformação do sistema solar para um modelo dinâmico.
    \item Adição ao modelo de um cometa, que segue a sua própria trajetória, definida numa curva \textit{Catmull-Rom}.
\end{itemize}

Um dos pontos mencionados, na forma de trabalho futuro, era relativo à evolução dos \textit{VBOs}
para permitirem o uso de índices. Na presente iteração permanece uma componente a adicionar, sendo
de alto valor na procura de uma solução otimizada congruentemente.
\end{document}