\documentclass[relatorio.tex]{subfiles}
\begin{document}
\section{Conclusões}
Esforço que permaneceu até a última fase do projeto, o grupo atualizou
ambos os programas, \textit{generator} e \textit{engine},
seguindo a arquitetura inicialmente estabelecida.
Deste modo, verifica-se a coesão das decisões tomadas inicialmente,
sendo apenas notável a evolução do projeto com a adição de novos elementos.

Não obstante as primitivas objetivo definidas em fases anteriores, nomeadamente
relativas ao uso de memória e na organização, foi o desejo do grupo 
aplicar todo o seu conhecimento agregado no decorrer do ano letivo.

É de notar nesta fase:
\begin{itemize}
    \item Criação dos vetores normais a cada vértice das figuras geométricas.
    \item Criação das coordenadas de texturas respetivas.
    \item Atualização do sistema solar, para um modelo texturado e com iluminação.
\end{itemize}

No seguimento do trabalho desenvolvido, o grupo pretendia acrescentar algumas funcionalidades,
nomeadamente a utilização de índices, acréscimo de algumas \textit{demos} mais elaboradas,
melhoria da interação do utilizador com o sistema (apresentação de um menu, possibilidade de 
carregar em múltiplas teclas em simultâneo), entre outros.
Não obstante, a equipa docente ter flexibilizado a entrega do trabalho, 
tais funcionalidades adicionais não foram cumpridos por falta de disponibilidade do grupo.
\end{document}