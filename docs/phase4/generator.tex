\documentclass[relatorio.tex]{subfiles}
\begin{document}

\subsection{Cálculo das Normais} \label{subsec:normals}
Para permitir a iluminação de cada figura geométrica 
criou-se uma nova instância da classe 
\mintinline{cpp}{t_points normals(number_vectors)}, 
para armazenar todos os vetores normais.

Reutilizou-se a classe \textit{t\_points},
representando cada \textbf{vetor} com
a classe \mintinline{cpp}{Point(x, y, z )}.
\dots apesar da nomenclatura não traduzir
diretamente, vértices e vetores têm o mesmo 
\textit{layout} para as suas coordenadas, 
evitando-se criar classes desnecessárias.

A variável \textit{number\_vectors} corresponde ao 
número total de \textbf{vetores} do objeto.
\dots uma vez que será necessário construir vetores 
para cada um vértice, corresponde ao número de 
vértices do objeto.

Deste modo, para cada figura geométrica:

\subsubsection{\textit{Plane}}

Um plano, ou \textit{plane}, é constituído por dois quadriláteros,
idênticos, um orientado para o lado positivo do eixo \textit{Y}
e outro orientado para o lado negativo deste mesmo eixo.

Pode-se afirmar que: \\
\textbf{Todos os vértices que se encontram na mesma superfície, 
têm a mesma normal.}

Pelo que:
\begin{eqnarray}
    v1 \epsilon plano_{1} \\
    v2\epsilon plano_{1} \implies normal_{v1} = normal_{v2}
\end{eqnarray}

Neste sentido, só existem \textbf{2 vetores normais únicos}.
Um para representar o plano, numa visão de cima,
e outro para o representar numa visão de baixo.

Dado a ser construído sobre o plano \textit{XZ}, 
estes dois vetores são estáticos:
\begin{eqnarray}
    vetorNormal_{plano superior} = (0,1,0) \\
    vetorNormal_{plano inferior} = (0,-1,0) 
\end{eqnarray}

\subsubsection{\textit{Box}}

Semelhante ao plano, a \textit{Box} ou caixa, 
tem \textbf{6 superfícies planas}.

Deste modo, só existem \textbf{6 vetores normais} 
diferentes para cada vértice da figura geométrica.

A caixa encontra-se desenhada na origem, centrada,
sendo que todas as suas faces são perpendiculares
aos eixos \textit{XYZ}(positivos e negativos).
Consequentemente, a normal a cada plano vai ser 
paralela aos eixos.

Deste modo, apresenta:
\begin{eqnarray}
    vetorNormal_{plano x positivo} = (1,0,0) \\
    vetorNormal_{plano x negativo} = (-1,0,0) \\
    vetorNormal_{plano y positivo} = (0,1,0) \\
    vetorNormal_{plano y negativo} = (0,-1,0) \\
    vetorNormal_{plano z positivo} = (0,0,1) \\
    vetorNormal_{plano z negativo} = (0,0,-1) \\
\end{eqnarray}

\subsubsection{\textit{Sphere}}
O cálculo das normais na esfera é feito com base na seguinte proprieada
geométrica:
Se p é um vértice da esfera:
$ p = (raio * sin(\alpha) * cos(\beta), raio * sin(\beta), raio * cos(\alpha) * cos(\beta))$
e como a esfera está centrada no ponto (0, 0, 0), o vetor normal é obtido através
da normalização do vetor n, que é igual a p - (0, 0, 0).

\subsubsection{\textit{Cone}}
O cálculo das normais do cone foram feitas com 2 abordagens diferentes,
para a base, uma vez que esta é desenhado sobre o plano XZ, consequentemente
a normal será a apontar para o Y negativo, isto é (0, -1, 0).

Para a lateral do cone, a normal foi obtida a partir do produto cartesiano
de um vetor tangente à face lateral do cone com 

\subsubsection{\textit{Cylinder}}
O cilindro é composto por 2 bases, sendo o a normal da base inferior o inverso
da normal da base superior. Como este é construído sobre o XZ, a normal da base superior
é (0, 1, 0) e da base inferior (0, -1, 0).

Para calcular as normais do parte lateral do cone, foi utilizada a seguinte propriedade
geométrica:

Se p é um vértica da face lateral do cilindro:
$ p = (raio * sin(\alpha), y, raio * cos(\alpha))$
então:
$ n = (sin(\alpha), 0, cos(\alpha))$

\subsubsection{\textit{Torus}}
As normais do \textit{torus} são calculadas de uma forma muito similar à das esferas,
tendo os pontos:

$ c = (raio * sin(\alpha) * cos(\beta), raio * sin(\beta), raio * cos(\alpha) * cos(\beta))$

$ p = (tamanho * sin(\alpha) * cos(\beta), tamanho * sin(\beta), tamanho * cos(\alpha) * cos(\beta))$

em que raio é o a distância do centro(0, 0, 0) ao centro do \textit{torus} e o tamanho é o raio interno
do \textit{torus}, então sabemos que $c + p$ e $c - p$ são vértice do \textit{torus} com normais inversas. 
Desta forma, é possível concluir que a normal destes vértices são obtidas, respetivamente, pela normalização 
do vetor obtido a partir subtração do ponto $p$ pelo ponto $c$ e da subtração do ponto $c$ pelo ponto $p$.

\subsection{Cálculo das Coordenadas de textura} \label{subsec:texCoord}
Foi necessário, para a implementação das texturas no modelo,
calcular as coordenadas de textura para cada figura geométrica.

Neste caso, estas coordenadas são \textit{2D}, pelo que
não era viável utilizar novamente a classe \textit{t\_points}.
Optou-se por utilizar um \textbf{vetor} de \textit{floats},
\mintinline{cpp}{vector<float> p_textures}.

Segue-se para cada figura geométrica:
\subsubsection{\textit{Plane}}

Cada plano é construído com um comprimento, \textit{l}, 
e número de divisões, \textit{div}.
Deste modo, para enquadrar nas coordenadas de textura,
que oscilam entre 0 e 1, foi necessário criar 
um \textit{step}.
\dots como tem o mesmo comprimento horizontal e 
verticalmente, apenas é criado um.

Este step é calculado como:
\begin{eqnarray}
    step = 1.0 / num_{divisoes}
\end{eqnarray}
\dots pelo que temos a translação da distância de uma
divisão no plano, para o eixo da textura a inserir.

Do mesmo modo que foram inseridos os vértices,
seguindo a \textbf{regra da mão direita}, 
as coordenadas de textura foram inseridas no seu \textit{vetor}.
\dots construíndo os triângulos do mesmo 
modo que estavam a ser inicialmente.

Efetua-se o mesmo processo para ambas as 
superfícies do plano, dado a serem 
idênticas.

\subsubsection{\textit{Box}}

No caso da caixa, dada à sua semelhança com o plano 
(é repetição deste em 6 orientações diferentes),
optou-se pela mesma estratégia.

A caixa apresenta, do mesmo modo, dois parâmetros: comprimento (\textit{length})
e divisões (\textit{divisions}), pelo que traduz-se 
diretamente do \textbf{plano}.

\subsubsection{\textit{Sphere}}

\subsubsection{\textit{Cone}}
Para coordenadas de textura da base do cone, o centro da base
corresponde ao centro da textura(0.5, 0.5) e cada slice corresponde
a um slice também na textura com o mesmo ângulo $\alpha$ mas com um raio
de 0.5.

Para a parte lateral foram criados 2 \textit{steps}:
\begin{itemize}
    \item $xStep = 1.0 / slices$
    \item $yStep = 1.0 / stacks$
\end{itemize}

A construção do cone é feita com base num ciclo que itera o número de \textit{stacks}
vezes em que o yStep começa a 0 e incremete yStep a cada iteração. Em que cada iteração d
este ciclo é percorrido outro ciclo com número de \textit{slices} iterações, eu que o xStep começa
a 0 e incrementa xStep a cada iteração. Com estas duas variavéis é facilmente obtido as coordenas de textura
de cada ponto.


\subsubsection{\textit{Cylinder}}
A estratégia utilizada no cilindro foi a mesma que no cone, sendo
as texturas das duas bases do cilindro obtidas da mesma forma que
a base do cone e o mesmo para as coordenadass de texturas da face lateral.

%\subsubsection{\textit{Torus}}


\subsection{\textit{Writer}} \label{subsec:writer}

Para poder inserir as duas novas \textit{tags}
dentro dos ficheiros \textit{.3D}, pequenas 
alterações ao \textit{writer} tiveram de ser feitas.

Para começar, todas as funções anteriormente 
criadas para construir cada uma das figuras 
geométricas foram alteradas.
Agora, retornam um tuplo com \textbf{3 valores}:
\begin{enumerate}
    \item Todos os vértices do objeto.
    \item Todos os vetores normais do objeto.
    \item Todas as coordenadas de textura para o objeto.
\end{enumerate}
\dots respetivamente.

Para clarificação, o tuplo apresenta-se como:
\begin{minted}{cpp}
    tuple<t_points, t_points, std::vector<float>> 
\end{minted}

Após essa alteração, só restou atualizar a função:
\begin{minted}{cpp}
    void write_xml(const char* filepath, GLenum type, t_points all_points, t_points all_normals, std::vector<float> texCoords)
\end{minted}
\dots recebe atualmente cada uma das listas de vértices/vetores 
do tuplo e escreve-as pela ordem estipulada.

Assim, preservou-se a estrutura previamente definida 
para a função \textit{int main(int argc, const char** argv)},
sendo feitos mínimos \textit{refactors}.

\subsubsection{Adicional} \label{subsec:aster}

Para popular o modelo com mais elementos decidiu-se 
adicionar uma nuvem de esferas de organização aleatória,
um cinturão de asteróides. 

A função que criava este modelo, conjunto de figuras:
\begin{minted}{cpp}
    std::tuple<t_points, t_points, std::vector<float>> create_asteroids(double distMin, double distMax, int maxSize, int slices, int stacks, double alphaMax, double betaMax, int numAsteroids)
\end{minted}
\dots sendo cada asteróide uma esfera.

Seria necessário indicar a :
\begin{itemize}
    \item \textbf{distMin}: A distância mínima a um asteroide;
    \item \textbf{distMax}: A distância máxima a um asteroide;
    \item \textbf{maxSize}: O raio máximo do asteroide;
    \item \textbf{slices}: O número de slices da esfera;
    \item \textbf{stacks}: O número de stacks da esfera;
    \item \textbf{alphaMax}: O valor máximo do ângulo alpha;
    \item \textbf{betaMax}: O valor máximo do ângulo beta;
    \item \textbf{numAsteroids}: O número de asteroides.
\end{itemize}

Para efetuar deste modo, foi necessário criar uma variação 
da figura geométrica \textbf{esfera}, a função:

\begin{minted}{cpp}
    std::tuple<t_points, t_points, std::vector<float>> create_sphere(int radius, int slices, int stacks, Point offset)
\end{minted}
\dots sendo adicionado um \textit{offset} aleatório compreendido
entre a distância mínima e máxima, para posicionar as esferas,
com base na variação de um ângulo $\alpha$, horizontal,
e um ângulo $\beta$, vertical.

\end{document}