\documentclass[relatorio.tex]{subfiles}
\begin{document}
    
\subsection{Cálculo das Normais} \label{subsec:normals}
Para permitir a iluminação de cada figura geométrica 
criou-se uma nova instância da classe 
\mintinline{cpp}{t_points normals(number_vectors)}, 
para armazenar todos os vetores normais.

Reutilizou-se a classe \textit{t\_points},
representando cada \textbf{vetor} com
a classe \mintinline{cpp}{Point(x, y, z )}.
\dots apesar da nomenclatura não traduzir
diretamente, vértices e vetores têm o mesmo 
\textit{layout} para as suas coordenadas, 
evitando-se criar classes desnecessárias.

A variável \textit{number\_vectors} corresponde ao 
número total de \textbf{vértices} do objeto, uma vez 
que será necessário construir vetores para cada um 
individualmente.

Deste modo, para cada figura geométrica:

\subsubsection{\textit{Plane}}
Como consequência de todas as superfícies serem planas, 
todos os vértices que se encontram no mesmo plano
têm o mesmo vetor normal.

Pelo que:
\begin{eqnarray}
    v1 \subset plano_{1} \\
    Se v2\subset plano_{1} \implies normal_{v1} = normal_{v2}
\end{eqnarray}

A figura geométrica é um único plano, 
logo só existem \textbf{2 vetores normais únicos}.
Um para representar o plano, numa visão de cima,
e outro para o representar numa visão de baixo.

Dado a ser construído sobre o plano \textit{XZ}, 
estes dois vetores são estáticos:
\begin{eqnarray}
    vetorNormal_{plano superior} = (0,1,0) \\
    vetorNormal_{plano inferior} = (0,-1,0) 
\end{eqnarray}

\subsubsection{\textit{Box}}

De modo análogo ao plano, a \textit{Box} ou caixa, 
tem \textbf{6 superfícies planas}.

Deste modo, só existem \textbf{6 vetores normais} 
diferentes para cada vértice da figura geométrica.


\subsubsection{\textit{Sphere}}

\subsubsection{\textit{Cone}}

\subsubsection{\textit{Cylinder}}

% \subsubsection{\textit{Torus}}

\subsection{Cálculo das Coordenadas de textura} \label{subsec:texCoord}

\subsubsection{Adicional}



\end{document}